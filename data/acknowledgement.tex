\newpage

\Acknowledgment{

    至此论文结束之际,心中感慨万千。两年半的研究生生活留下了很多美好而难忘的回忆与经历。如今,在这2018年即将过去的时刻,心中除了对校园生活的怀念和不舍,心中充满了对所有老师和同学的感激。

    首先,我要感谢江建军教授。江老师作为IEI实验室的负责人,一直孜孜不倦地为实验室的建设和发展而付出。正是在江老师的带领下,IEI才有了现在的规模和软硬件条件。在整个研究生期间,江老师教导我们要树立正确的科研观、要把专业理论基础学牢。还通过组织观看我国空军发展的纪录片,激励了实验室全体人员为祖国的雷达隐身事业贡献自己的一份力量。同时,将老师还重视实验室的体育运动,鼓励大家参与体育运动。

    \shapepar{\heartshape}然后,我要由衷地感谢我的导师缪灵副教授,感谢缪老师对我的知遇之恩,在加人IEI大家庭之后,缪老师尊重我的研究兴趣和擅长的方向,给我规划了科研方向,才让我没有在科研的道路迷失方向。在科研中,缪老师是一个宽容大度的人,他不会因为我的一次失误而对我失去信心,但缪老师也是一个讲原则的人,如果安排的任务三次都没有完成,他也会毫不留情,和缪老师相处的这些日子使我养成了按时完成任务的习惯。在生活中,缪老师鼓励我参加组会,他经常说:只有在组会中讲出来你做的事情,才能有所改进。缪老师的学习能力很强,思路很开阔,虽然他已经多年不写代码,但是他扎实的基本功和活跃的思维方式让我羡慕不已。

    接着我要感谢别少伟副教授和贺云博士后。别老师严谨的学术态度和刻苦钻研的科研精神让我敬佩不已。贺博士后,虽然很年轻,但是不论是他的科研能力还是对实验室的管理能力都很出色。在今后的生活中,我会以别老师和贺博士后为榜样,认真做事。

    \shapepar{\starshape}感谢张玉禄、郦程丽、魏剑锋、吴松、查大册、郭赛、曹昭旺、李芮、张宝、吉晓、刘佳、徐葵、赵勇等博士生在科研方法上的指导。感谢金湾湾、徐壮、张宇豪、方博、孙文钊、许迎东、孙梦、夏靖、何凡、李莎、刘羽函、刘雨彤、雷志朋、汪杰鸣、万玫玉等硕士生在科研工作中的帮助。希望IEI在你们的努力下发展得越来越好!

    我还要感谢我的家人和关心我的人,没有他们在背后的支持和关怀,就没有今天的我。希望我以后能够给与你们回报,让家里人生活得更美好!

}