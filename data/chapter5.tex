\newpage

\section{总结与展望}

\subsection{总结}

本文在微波集成设计平台(MIDIS)的基础上,通过任务管理模块、数据库系统、有源无源FSS结构的数据处理、二维解空间、规律挖掘、结构优化等功能模块,完成了基于数据库系统的频率选择表面吸波结构的数据分析与结构优化。本文的主要研究成果如下:

(1)限于对课题组计算资源进行智能管理的需求,开发了任务管理模块。该模块能够对计算资源进行自动分配,对计算任务进行集中管理。以MIDIS和核心,形成了任务管理、数据遍历、智能优化、数据库存储、数据处理、数据挖掘的框架,对MIDIS与HFSS和CST之间的接口调用和参数化方式进行了介绍,特别针对CST的一些特殊设置进行了说明。

(2)基于对课题组计算数据的安全和数据分析的快捷性考虑,使用了企业级数据库Mongodb,设计了数据库系统。以数据库为基础,开发了针对无源仿真数据处理模块和有源数据处理模块以及利用降维技术的多种变量分析工具。数据库系统的实现,极大提高了对多维变量的吸波结构的分析效率。使用单圆环FSS结构,分别设计了无源吸波体,并对该结构的变量空间进行遍历,使用相应工具得到需要的结构模型,用降维分析工具对单圆环FSS结构进行了变量分析。通过单圆环的改进设计了有源FSS结构,并计算了不同状态的结构反射率曲线,通过包络线处理的方式,得到有源单圆环FSS结构的吸波性能。提出了针对寄生参数带来的有源仿真数据与测量数据不吻合的解决方法,即通过修改模型,对寄生参数继续遍历拟合的方式得到寄生参数的值。

(3)介绍了代理模型优化算法贝叶斯优化,分析了该算法原理和用来优化FSS吸波结构的可行性,并开发了针对无源FSS结构的贝叶斯优化模块。设计了单方环FSS吸波结构,分别使用遗传算法和贝叶斯优化在4-8GHz对吸波结构进行优化,两个算法都能优化出在4-8GHz达到-10dB的吸波结构,但是贝叶斯优化的效率要高于遗传算法。通过对贝叶斯优化的数据进行分析,变量的分布疏密的不同和性能分布区域不同进一步表明了算法对FSS结构优化的有效性。

\subsection{展望}

本文实现了利用数据库系统对频率选择表面吸波结构进行数据分析和结构优化,但是在以下一些方面还值得进一步研究:

(1) 本文使用了较为简单的FSS吸波结构图案,计算的数据量也较少。可以对同种拓扑结构的图案进行大量计算和存储,然后对同一种类型的FSS结构进行数据分析,这样就可以得到更为普遍的规律和设计指导。也可以使用现代的数据分析技术,如神经网络,聚类算法、分类算法等对数据进行训练学习,提供吸波结构的性能预测等功能。

(2) 本文对有源吸波结构的仿真数据与测量数据的不吻合提供的解决方案较为简单,可以利用机器学习的算法对仿真与测量数据进行监督学习,提供一种具有普适性的针对仿真模型的校正方案。

(3) 本文设计了贝叶斯优化算法在频率选择表面上面的优化属于一种新的尝试,由于时间关系并没有使用多种不同的模型结构进行优化测试,只设计了单方环结构,可以设计不同特性的频率选择表面结构进行测试验证,通过优化结果调整算法的一些参数,使得该算法能够真正在频率选择表面的优化上产生较好的效果。
