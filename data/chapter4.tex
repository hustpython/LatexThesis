\newpage

\section{贝叶斯优化在频率选择表面中的应用}

近年来,为了提升FSS的优化设计效率,有很多优化方法被提出来,如遗传算法,支持向量回归算法,和模型代理法等。遗传算法是一种典型的全局优化算法,算法的规则简单并广泛应用于优化设计。然而,随着对FSS模型计算成本的增加,传统的遗传算法的收敛速度较慢,全局搜索能力也不突出,算法设置复杂,很多参数的设置需要依靠经验。模型代理法依靠统计学方法,通过采样的方法确定下一个个体[45-46]。不同于进化算法的种群迭代模式,贝叶斯优化算法每次只会产生一个新样本点,这样会在很大程度减少了计算量减少了设计周期。

本章将会对贝叶斯优化算法的原理进行介绍,并且设计了基于贝叶斯优化算法的无源FSS优化模块。最后分别使用贝叶斯优化算法和遗传算法对无源单方环FSS进行优化,并根据优化结果比较两组优化算法的异同。

\subsection{贝叶斯优化算法}

贝叶斯优化算法(BOA)是一种用于优化“黑盒问题”的优化策略,这里的“黑盒问题”指的是优化的目标的具体函数形式是未知的,无法利用数学上的求导公式找到其极值点。比如在做一份糕点时,会考虑各种配料的用量和比例才能做出美味可口的蛋糕。在这样的需要多维度的场景当中,每个维度的可候选空间也非常大,此外各个维度属性之间可能相互影响,通过枚举遍历的方法进行尝试需要很大的时间成本和物质成本。在这种情况下,贝叶斯优化算法在较少的迭代次数下能够找到需要的参数。其算法流程如图4-1所示。