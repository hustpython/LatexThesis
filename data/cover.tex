% 分类号相关
{
\zihao{4}\selectfont\huawenzhongsong
\begin{flushleft}
\begin{tabular}{rl}
	{\makebox[5em][s]{分类号}}&\underline{\hskip 1.35cm \qquad}\\
	{\makebox[5em][s]{学校代码}}&\underline{\hskip 0.5cm 10487 \hskip 0.5cm} \\ 
\end{tabular}
\hfill %自动填充使得右边文字居右
\begin{tabular}{rl}
	{学号}&\underline{M201672235}\\
	
	{密级}&\underline{\hskip 2.7cm} \\ 
\end{tabular}
\end{flushleft}
}

\vskip 0cm
\begin{center}
	\includegraphics[width=8.63cm,height=1.8cm]{hustcoverlogo}% 封面华中科技大学logo
\end{center}
  
% 硕士学位论文
{\fontsize{53pt}{53pt}\selectfont\huawenzhongsong
\begin{center}
	{
	\renewcommand{\CJKglue}{\hskip 10pt} %字间隔10pt
	\textbf{硕士学位论文}}
\end{center}}
% 间隔1cm
\vskip 0cm
% 论文题目
{\fontsize{28pt}{28pt}\selectfont\lxk
\begin{center}
    %居中自动换行
	\setlength{\baselineskip}{50pt} %设置行间距
	\setlength\hsize{1\hsize}\centering %弹性长度
	{基于数据库系统的频率选择表面结构分析与优化}
\end{center}}
%空0cmm间隔
\vskip 0cm
% 设置学位信息的字体格式
% {}之间用于限制作用域
{\fontsize{20pt}{20pt}\selectfont\huawenzhongsong
{% 右移动1cm
\hspace{1cm}
% 设置表格中行距,下面的信息通过表格实现两端对齐
\vskip 1cm
\begin{tabular}{rl}
	{\makebox[6em][s]{学位申请人:}}&余忠伟\\
	
	{\makebox[6em][s]{学科专业:}}&软件工程 \\ 

    {\makebox[6em][s]{指导老师:}}&缪灵 \\ 
   
	{\makebox[6em][s]{答辩日期:}}& 2019年01月15日\\ 

\end{tabular}}
}
