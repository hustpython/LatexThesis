\newpage

\Cnabstract
{
通过吸波材料降低雷达散射截面积(RCS)是实现雷达隐身的重要举措之一,频率选择表面(FSS)作为一种新型结构型吸波材料,在RCS缩减方面发挥着重要的作用。影响FSS电磁性能的结构参数众多,如何对FSS吸波结构的参数之间进行有效分析、设计出满足特定目标频段的FSS结构已经成为学者关注的热点。本文以微波智能设计集成平台(MIDIS)为基础,研究完成了频率选择表面的结构分析与智能优化。

本研究开发了MIDIS平台的任务管理、数据库系统、贝叶斯优化算法等功能模块。完成了无源单圆环FSS结构的分析、有源单圆环FSS结构的分析、单方环FSS吸波结构的优化和设计。

任务管理模块是通过建立任务队列的方式将MIDIS平台上的计算任务进行集中管理。任务管理模块的实现,减少了创建任务的时间、提高了计算资源的利用效率。针对FSS仿真数据的特点,设计了数据库系统,该数据库系统能够对所有仿真数据进行快速存储与查询,为其他数据分析和优化算法模块的设计提供了数据接口。

基于数据库系统,完成了无源FSS吸波结构的和有源FSS吸波结构数据处理模块。通过降维技术,分析了该结构变量之间的变化对吸波性能的影响。对有源FSS结构的数据,通过指定有源器件对应的仿真变量,进行包络线处理,得到有源FSS结构对应的反射率曲线,针对有源FSS吸波结构的仿真结果与测量结果不吻合的问题,提出了修正模型,拟合器件寄生参数的方案。

本文还提出了一种基于模型代理的优化算法,贝叶斯优化算法(BOA),它通过概率统计学的方法创建函数分布模型,在模型中寻找最优解的位置,极大地提高了优化效率。为了比较该算法与遗传算法的效率,设计了单方环FSS吸波结构,优化目标频段为4-8GHz,结果显示两种算法都实现了优化目标,但BOA的优化时间更短。
        
}
\Cnkeywords{频率选择表面 \ 智能算法 \ 数据库 \ 吸波材料 \ 贝叶斯优化算法}

